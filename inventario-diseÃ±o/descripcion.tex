\documentclass{article}
\usepackage[utf8]{inputenc}
\usepackage{amsmath} % Para entornos matemáticos si fueran necesarios
\usepackage{amssymb} % Para símbolos matemáticos si fueran necesarios
\usepackage{amsthm}  % Para teoremas si fueran necesarios
%\usepackage{hyperref} % Para enlaces y referencias
%\usepackage{enumitem} % Para listas personalizadas
%\usepackage{xcolor} % Para colores si se desean (opcional)
%\usepackage{graphicx} % Para imágenes si se desean (opcional)
%\usepackage{geometry} % Para configurar márgenes
%\geometry{a4paper, margin=1in}

\title{Plan de Trabajo: Sistema de Inventario para Bebidas y Desayunos}
\author{Tu Nombre y Nombre del Colaborador}
\date{\today}

\begin{document}
	
	\maketitle
	
	\begin{abstract}
		Este documento presenta un plan de trabajo estructurado para el desarrollo de un sistema de inventario modular y de bajo acoplamiento, enfocado en el negocio de bebidas y desayunos. Se detallan los conceptos clave, la caja de herramientas de Rust, patrones de diseño, consideraciones de seguridad, testing y una propuesta de demo mínima viable.
	\end{abstract}
	
	\tableofcontents
	
	\section{Introducción}
	El objetivo es desarrollar un sistema de inventario robusto que permita gestionar productos, ingredientes, recetas, ventas y empleados para un negocio de bebidas y desayunos. Este plan prioriza la \textbf{modularidad} y el \textbf{bajo acoplamiento} desde las primeras fases del desarrollo.
	
	---
	
	\section{Conceptos Clave del Sistema (Alto Nivel)}
	Para este sistema, los conceptos fundamentales giran en torno a la gestión eficiente de recursos y la operativa de un negocio de alimentos y bebidas:
	
	\begin{itemize}[label=$\bullet$]%
%		\item \textbf{Gestión de Inventario Granular}: No solo controlar productos finales, sino los \textbf{ingredientes} que los componen, lo que implica un modelo de datos robusto para \textbf{productos base} y \textbf{recetas}.
		%\item \textbf{Automatización de Descuentos}: El sistema debe manejar automáticamente la baja de múltiples ingredientes del inventario al vender una \textbf{receta}.
		\item \textbf{Alertas Proactivas}: Capacidad de notificar sobre la \textbf{baja existencia} de productos, un pilar para evitar la falta de stock y pérdidas de ventas.
		\item \textbf{Roles y Permisos}: Diferenciar las capacidades de los usuarios (\textbf{empleados}) (ej., administradores, personal de inventario, cajeros) para asegurar la integridad de los datos y la seguridad.
		\item \textbf{Registro de Transacciones}: Mantener un \textbf{historial} de ventas y movimientos de inventario para auditoría y análisis.
		\item \textbf{Persistencia de Datos}: La información del inventario, ventas y empleados debe \textbf{guardarse} de forma segura y \textbf{cargarse} al reiniciar la aplicación.
	\end{itemize}
	
	---
	
	\section{Caja de Herramientas de Rust Esencial}
	Aquí están las herramientas y conceptos específicos de Rust que utilizaremos para construir un sistema robusto y modular:
	
	\begin{itemize}[label=$\bullet$]
%		\item \textbf{Modularidad y Bajo Acoplamiento} (\textbf{Traits} \& \textbf{Módulos}):
		%\begin{itemize}
		%	\item \textbf{Módulos (\texttt{mod})}\footnote{\href{https://doc.rust-lang.org/book/ch07-00-namespaces-to-organize-code.html}{The Rust Programming Language: Modules}}: Organizaremos el código en módulos lógicos (ej., \texttt{domain}, \texttt{repositories}, \texttt{services}, \texttt{cli}).
			%\item \textbf{Traits}\footnote{\href{https://doc.rust-lang.org/book/ch10-02-traits.html}{The Rust Programming Language: Traits}}: Fundamentales para definir \textbf{interfaces} y lograr bajo acoplamiento. Por ejemplo, \texttt{trait ProductRepository} que define cómo interactuar con los datos de productos, permitiendo múltiples implementaciones (JSON, en memoria para tests, futura base de datos).
		%\end{itemize}
		%\item \textbf{Estructuras de Datos} (\texttt{struct} y \texttt{enum})\footnote{\href{https://doc.rust-lang.org/book/ch05-00-structs.html}{The Rust Programming Language: Structs}, \href{https://doc.rust-lang.org/book/ch06-00-enums.html}{The Rust Programming Language: Enums}}: Modelaremos todas las entidades del negocio (\texttt{Producto}, \texttt{Receta}, \texttt{Venta}, \texttt{Empleado}, etc.).
		%\item \textbf{Manejo de Errores} (\texttt{Result<T, E>}, \texttt{?})\footnote{\href{https://doc.rust-lang.org/book/ch09-02-recoverable-errors-with-result.html}{The Rust Programming Language: Recoverable Errors with Result}}: Utilizaremos el sistema de errores de Rust para gestionar fallos de forma explícita y segura. Podríamos definir tipos de error personalizados para el dominio. Considerar \textbf{\texttt{thiserror}}\footnote{\href{https://crates.io/crates/thiserror}{Crates.io: thiserror}} y \textbf{\texttt{anyhow}}\footnote{\href{https://crates.io/crates/anyhow}{Crates.io: anyhow}} para manejo avanzado.
		%\item \textbf{Serialización/Deserialización} (\textbf{\texttt{serde}}\footnote{\href{https://crates.io/crates/serde}{Crates.io: serde}}): Para guardar y cargar nuestros datos de objetos en archivos (inicialmente JSON, con \textbf{\texttt{serde\_json}}\footnote{\href{https://crates.io/crates/serde_json}{Crates.io: serde_json}}).
		%\item \textbf{Colecciones de Datos} (\textbf{\texttt{HashMap}}\footnote{\href{https://doc.rust-lang.org/std/collections/struct.HashMap.html}{Rust Standard Library: HashMap}}, \textbf{\texttt{Vec}}\footnote{\href{https://doc.rust-lang.org/std/vec/struct.Vec.html}{Rust Standard Library: Vec}}): Para almacenar y acceder eficientemente a nuestros productos, recetas y empleados. \texttt{HashMap} será clave para búsquedas rápidas por ID.
		%\item \textbf{Fechas y Horas} (\textbf{\texttt{chrono}}\footnote{\href{https://crates.io/crates/chrono}{Crates.io: chrono}}): Para registrar timestamps en ventas y gestionar gastos fijos.
		%\item \textbf{Configuración}: Uso de \textbf{\texttt{config}}\footnote{\href{https://crates.io/crates/config}{Crates.io: config}} para cargar configuraciones y %\textbf{\texttt{dotenv}}\footnote{\href{https://crates.io/crates/dotenv}{Crates.io: dotenv}} para variables de entorno en desarrollo.
		%\item \textbf{Registro (Logging)}: \textbf{\texttt{log}}\footnote{\href{https://crates.io/crates/log}{Crates.io: log}} como interfaz y \textbf{\texttt{env\_logger}}\footnote{\href{https://crates.io/crates/env_logger}{Crates.io: env_logger}} (o \textbf{\texttt{tracing}}\footnote{\href{https://crates.io/crates/tracing}{Crates.io: tracing}}) para la implementación.
	\end{itemize}
	
	---
	
	\section{Patrones de Diseño (Alto Nivel)}
	Aunque no aplicaremos patrones complejos como CQRS o Event Sourcing de entrada, usaremos patrones fundamentales para la modularidad:
	
	\begin{itemize}[label=$\bullet$]
		%\item \textbf{Capas Arquitectónicas}: Dividiremos la aplicación en capas claras:
	%	\begin{itemize}
		%	\item \textbf{Dominio}: Contiene la lógica de negocio central (reglas, entidades).
			\item \textbf{Servicios (Aplicación)}: Orquestra las operaciones de negocio, utilizando la lógica del dominio y los repositorios.
			\item \textbf{Repositorios (Infraestructura)}: Encapsula la lógica de persistencia de datos (cómo se guardan y cargan las entidades).
			\item \textbf{Presentación (CLI)}: La interfaz de usuario que interactúa con los servicios.
		\end{itemize}
		\item \textbf{Inyección de Dependencias (mediante \texttt{traits})}: Los servicios dependerán de \texttt{traits} de repositorio, no de implementaciones concretas, permitiendo flexibilidad y facilidad de prueba.
	\end{itemize}
	
	---
	
	\section{Seguridad con Hashing en Login}
	\begin{itemize}[label=$\bullet$]
		%\item \textbf{Contraseñas Hasheadas}: Almacenaremos las contraseñas de los empleados usando funciones de hashing seguras (ej., Argon2 o Bcrypt si usamos una crate como %\textbf{\texttt{argon2rs}}\footnote{\href{https://crates.io/crates/argon2rs}} o \textbf{\texttt{bcrypt}}\footnote{\href{https://crates.io/crates/bcrypt}}, o SHA-256 como primera aproximación para entender el concepto). \textbf{Nunca guardarlas en texto plano.}
	%	\item \textbf{Roles}: Implementaremos un sistema de roles simple para la \textbf{autorización}, definiendo qué acciones puede realizar cada tipo de empleado.
	\end{itemize}
	
	---
	
	\section{Concurrencia de Datos (Consideración Futura)}
	Para el demo inicial, la aplicación será \textbf{monohilo} y operará de forma secuencial. Si en el futuro el sistema necesita manejar múltiples usuarios simultáneamente o tareas en segundo plano, exploraremos conceptos como:
	
	\begin{itemize}[label=$\bullet$]
%		\item \textbf{Hilos (\texttt{std::thread})}\footnote{\href{https://doc.rust-lang.org/std/thread/index.html}{Rust Standard Library: thread}}: Para ejecutar código en paralelo.
	%	\item \textbf{Mutex / RwLock}\footnote{\href{https://doc.rust-lang.org/std/sync/struct.Mutex.html}{Rust Standard Library: Mutex}, \href{https://doc.rust-lang.org/std/sync/struct.RwLock.html}{Rust Standard Library: RwLock}}: Para gestionar el acceso seguro y concurrente a los datos compartidos en memoria.
%		\item \textbf{Programación Asíncrona} (\textbf{\texttt{tokio}}\footnote{\href{https://crates.io/crates/tokio}{Crates.io: tokio}} / \textbf{\texttt{async-std}}\footnote{\href{https://crates.io/crates/async-std}{Crates.io: async-std}}): Para operaciones de I/O que no bloqueen el hilo principal (ej., si se conecta a una base de datos o API externa).
%	\end{itemize}
	
	---
	
	\section{Pruebas}
	\begin{itemize}[label=$\bullet$]
	%	\item \textbf{Pruebas Unitarias}\footnote{\href{https://doc.rust-lang.org/book/ch11-01-writing-tests.html}{The Rust Programming Language: Writing Tests}}: Para verificar que las funciones individuales y la lógica de negocio (ej., el descuento de ingredientes) funcionan correctamente.
	%	\item \textbf{Pruebas de Integración}\footnote{\href{https://doc.rust-lang.org/book/ch11-03-test-organization.html#integration-tests}{The Rust Programming Language: Integration Tests}}: Para asegurar que los diferentes módulos (servicios, repositorios) se comunican y operan bien en conjunto.
	%	\item \textbf{Mocks/Fakes}: Utilizaremos implementaciones en memoria de nuestros \texttt{traits} de repositorio para aislar la lógica de negocio durante las pruebas unitarias.
	\end{itemize}
	
	---
	
	\section{Demo Básica: Inventario y Alertas}
	El objetivo del demo es mostrar la \textbf{funcionalidad central} de control de stock y aviso.
	
	\begin{itemize}[label=$\bullet$]
	%	\item \textbf{Funcionalidades del Demo}:
	%	\begin{itemize}
		%	\item \textbf{Registro de Productos}: Permitir añadir nuevos productos con nombre, cantidad inicial y un umbral de \texttt{stock\_minimo\_alerta}.
			\item \textbf{Listado de Inventario}: Mostrar el estado actual de todos los productos en el inventario.
			\item \textbf{Venta Simplificada de Recetas}:
			\begin{itemize}
				\item Tener \textbf{2-3 recetas predefinidas} (ej. "Café Simple", "Jugo de Naranja").
				\item Al "vender" una receta, el sistema descuenta los ingredientes correspondientes del inventario.
				\item \textbf{Validación de Stock}: Si no hay suficientes ingredientes, la venta no procede y se notifica al usuario.
			\end{itemize}
			\item \textbf{Alertas de Bajo Stock}: Mostrar una lista de productos que han alcanzado o superado su \texttt{stock\_minimo\_alerta}.
			\item \textbf{Persistencia}: Guardar y cargar el inventario en un archivo JSON para que los cambios persistan.
		\end{itemize}
		\item \textbf{Interfaz}: Una aplicación de \textbf{línea de comandos (CLI)} simple con un menú de opciones (ej. "1. Ver Inventario", "2. Registrar Venta", "3. Ver Alertas", "4. Salir").
	\end{itemize}
	
	---
	
	\section{Plan de Trabajo Detallado (Para ti y tu Colaborador)}
	Este plan está diseñado para ser iterativo, construyendo funcionalidades paso a paso.
	
	\subsection{Fase 1: Configuración y Modelado Básico (1-2 días)}
	%\begin{enumerate}[label=\arabic*.]
	%	\item \textbf{Inicializar Proyecto Rust}: Crear un nuevo proyecto (\texttt{cargo new inventory\_system}).
	%	\item \textbf{Estructura de Carpetas/Módulos}:
		\begin{itemize}
			\item \texttt{src/main.rs} (punto de entrada, CLI)
			\item \texttt{src/domain/mod.rs} (definición de structs: \texttt{Producto}, \texttt{IngredienteReceta}, \texttt{Receta}, \texttt{Error})
			\item \texttt{src/repositories/mod.rs} (definición de \texttt{traits} de repositorio)
			\item \texttt{src/services/mod.rs} (definición de \texttt{traits} de servicios)
		\end{itemize}
		%\item \textbf{Definir Entidades del Dominio}:
		\begin{itemize}
			\item Crea los \texttt{structs} para \texttt{Producto} y \texttt{Receta} con sus campos esenciales (ID, nombre, cantidad/ingredientes, stock mínimo).
			\item Define un \texttt{enum} básico para \texttt{Error} con al menos \texttt{NotFound} y \texttt{InsufficientStock}.
		\end{itemize}
		%\item \textbf{Configuración Básica}: Utiliza la crate \texttt{config} para cargar la ruta del archivo de datos JSON y \texttt{dotenv} para variables de entorno en desarrollo.
	%	\item \textbf{Logging Básico}: Integra \texttt{log} y \texttt{env\_logger} para mensajes informativos.
	%\end{enumerate}
	
	\subsection{Fase 2: Persistencia y CRUD de Inventario (2-3 días)}
	%\begin{enumerate}[label=\arabic*.]
	%	\item \textbf{Trait \texttt{ProductRepository}}: Define el \texttt{trait} en \texttt{src/repositories/mod.rs} con métodos para \texttt{get\_by\_id}, \texttt{save}, \texttt{get\_all}, etc.
	%	\item \textbf{Implementación JSON de \texttt{ProductRepository}}: Crea un \texttt{struct JsonProductRepository} que implemente el \texttt{trait} y use \texttt{serde\_json} para leer/escribir del archivo JSON.
		%\item \textbf{Servicio de Inventario (\texttt{InventoryService})}:
		\begin{itemize}
			\item Crea un \texttt{struct InventoryService} en \texttt{src/services/mod.rs} que tome una instancia de \texttt{dyn ProductRepository} (mediante \texttt{Box<dyn ProductRepository>}) en su constructor.
			\item Implementa métodos para \textbf{añadir productos} y \textbf{listar todos los productos}.
		\end{itemize}
		%\item \textbf{CLI Básica}: En \texttt{src/main.rs}, crea un bucle de menú que permita:
		\begin{itemize}
			\item Cargar el inventario desde el archivo al inicio.
			\item Añadir un nuevo producto.
			\item Listar el inventario actual.
			\item Guardar los cambios al salir.
		\end{itemize}
	%\end{enumerate}
	
	\subsection{Fase 3: Recetas y Lógica de Descuento (3-4 días)}
%	\begin{enumerate}[label=\arabic*.]
	%	\item \textbf{Definir \texttt{Receta} y \texttt{IngredienteReceta}}: Asegúrate de que tu \texttt{struct Receta} en \texttt{src/domain/mod.rs} pueda contener una lista de \texttt{IngredienteReceta}.
%		\item \textbf{Trait \texttt{RecipeRepository}}: Similar a \texttt{ProductRepository}, define un \texttt{trait} para gestionar recetas.
%		\item \textbf{Implementación JSON de \texttt{RecipeRepository}}: Crea un \texttt{struct JsonRecipeRepository}.
%		\item \textbf{Lógica de Dominio para Descuento}: En \texttt{src/domain/mod.rs}, implementa una función (o método asociado a \texttt{Receta}) que:
		\begin{itemize}
			\item Tome un \texttt{HashMap<String, Producto>} (el inventario) y una \texttt{Receta}.
			\item Verifique si hay suficiente stock de \textbf{todos los ingredientes}.
			\item Si hay, descuente las cantidades necesarias del inventario.
			\item Retorne un \texttt{Result} indicando éxito o un error de \texttt{InsufficientStock}.
		\end{itemize}
		%	\item \textbf{Actualizar \texttt{InventoryService} (o crear \texttt{SaleService})}:
		\begin{itemize}
			\item Añade un método al \texttt{InventoryService} (o crea un nuevo \texttt{SaleService} si prefieres una separación más estricta para ventas) que:
			\item Tome una \texttt{Receta} y una cantidad.
			\item Use la lógica de dominio para descontar los ingredientes.
			\item Actualice el inventario usando el \texttt{ProductRepository}.
		\end{itemize}
		%\item \textbf{Actualizar CLI para Venta}:
		\begin{itemize}
			\item Añade una opción para "Vender Receta".
			\item Permite al usuario elegir una receta (ej., por su ID) y una cantidad.
			\item Muestra el resultado de la venta (éxito o error por falta de stock).
		\end{itemize}
	%\end{enumerate}
	
	\subsection{Fase 4: Alertas y Demo Final (2-3 días)}
	%\begin{enumerate}[label=\arabic*.]
		%\item \textbf{Lógica de Alertas}: En \texttt{src/domain/mod.rs} o en \texttt{InventoryService}, implementa una función para \texttt{obtener\_productos\_bajo\_stock()}.
		%\item \textbf{Integrar Alertas en CLI}:
		\begin{itemize}
			\item Añade una opción de menú para "Ver Alertas de Bajo Stock".
			\item Al inicio del programa (después de cargar el inventario), muestra un resumen si hay alertas.
		\end{itemize}
	%	\item \textbf{Pruebas Básicas}:
		\begin{itemize}
			\item Escribe pruebas unitarias para la lógica de descuento de ingredientes.
			\item Escribe una prueba de integración para el flujo de añadir producto y listarlo.
		\end{itemize}
%		\item \textbf{Refinamiento y Demo}:
		\begin{itemize}
			\item Asegúrate de que los mensajes en la CLI sean claros.
			\item Prepara datos iniciales en tu archivo JSON para una demostración fluida.
			\item Practica el flujo de la demo.
		\end{itemize}
%	\end{enumerate}
	
\end{document}